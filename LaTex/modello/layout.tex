%\documentclass[a4paper, oneside, openany]{book}
\documentclass{article}

%**************************************************************
% Importazione package
%************************************************************** 

% Permette di resettare il contatore delle note a piè di pagina ad ogni pagina
\usepackage[perpage, bottom]{footmisc}

% permette di modificare i margini
%\usepackage[top=3.1cm, bottom=3.1cm, left=2.2cm, right=2.2cm]{geometry}
\usepackage[a4paper]{../../modello/geometry}

% specifica con quale codifica bisogna leggere i file
\usepackage[utf8]{inputenc}

% necessario per risolvere il problema del carattere invisibile per l'a capo
\DeclareUnicodeCharacter{00A0}{ }

% per scrivere in italiano e in inglese;
% l'ultima lingua (l'italiano) risulta predefinita
\usepackage[english, italian]{babel}

% imposta lo stile italiano per i paragrafi
\usepackage{parskip}

% fornice elenchi numerati in ordine inverso
\usepackage{etaremune}

% comandi per l'appendice
\usepackage{appendix}
\renewcommand\appendixtocname{Appendici}
\renewcommand{\appendixpagename}{Appendici}

% numera anche i paragrafi
\setcounter{secnumdepth}{6}

% elenca anche i paragrafi nell'indice
\setcounter{tocdepth}{6}

% permetti di definire dei colori
\usepackage[usenames,dvipsnames]{color}

% permette di usare il comando "paragraph" come subsubsubsection!
\usepackage{titlesec}

% permette di inserire le immagini/tabelle esattamente dove viene usato il
% comando \begin{figure}[H] ... \end{figure}
% evitando che venga spostato in automatico
\usepackage{float}

% permette l'inserimento di url e di altri tipi di collegamento
\usepackage[colorlinks=true]{hyperref}

\hypersetup{
    colorlinks=true, % false: boxed links; true: colored links
    citecolor=black,
    filecolor=black,
    linkcolor=black, % color of internal links
    urlcolor=Maroon  % color of external links
}

% permette al comando \url{...} di andare a capo a metà di un link
% COMMENT: non so se conviene usarlo, non è tanto bello un link spezzato
\usepackage{breakurl}

% immagini
\usepackage{graphicx}

% permette di riferirsi all'ultima pagina con \pageref{LastPage}
\usepackage{lastpage}

% tabelle su più pagine
\usepackage{longtable}

% per avere dei comandi in più da poter usare sulle tabelle
\usepackage{booktabs}

% tabelle con il campo X per riempire lo spazio rimanente sulla riga
\usepackage{tabularx}

% multirow per tabelle
\usepackage{multirow}

% colore di sfondo per le celle
\usepackage[table]{xcolor}
%\rowcolors{2}{gray!25}{} % Colora righe alterne in grigio (non funziona bene) 

% permette di fare longtable larghe tutta la pagina (parametro x)
% su Ubuntu non si può installare il pacchetto, deve essere in modello/
\usepackage{../../modello/tabu}

% imposta lo spazio tra le righe di una tabella
\setlength{\tabulinesep}{6pt}

% definisci un nuovo tipo di colonna P che permette di andare a capo con \newline
% e giustificata a sinistra
\usepackage{array}
\usepackage{ragged2e}
\newcolumntype{P}[1]{>{\RaggedRight \hspace{0pt}}p{#1}}

% personalizza l'intestazione e piè di pagina
\usepackage{fancyhdr}

% permette di inserire caratteri speciali
\usepackage{textcomp}

% permette di aggiustare i margini e centrare tabelle e figure
\usepackage{changepage}

%Permette di includere i grafici a barre
%IMPORTANTE: deve essere caricato prima di /pgfgantt altrimenti causa conflitto
\usepackage{pgfplots}

% permette di includere i diagrammi Gantt
% su Ubuntu non si può installare il pacchetto, deve essere in modello/
\usepackage{../../modello/pgfgantt}

% permette di includere i grafici a torta
\usepackage{../../modello/pgf-pie}

% necessario per pgf-pie
\usepackage{tikz}

% permette i path delle immagini con gli spazi
\usepackage{grffile}

% ruota le immagini
\usepackage{rotating}

% permetti di calcolare le larghezze facendo calcoli
\usepackage{calc}

\fancypagestyle{plain}{	
	% cancella tutti i campi di intestazione e piè di pagina
	\fancyhf{}
	\lhead{
		\includegraphics[height=1.5cm, width=1.5cm, keepaspectratio=true]{../../modello/or-bit_bkg.png}
		\parbox[b]{10cm}{
			\emph{\GroupName{}} \vspace{0pt} \\
			\emph{Progetto \ProjectName{}} \vspace{7pt}
		}
	}
	\chead{}
	%\rhead{
	%	\slshape \leftmark
	%}
	% Stampa la sezione in alto a destra sull'header
	\rhead{	
		\includegraphics[height=1.25cm, width=1.25cm, keepaspectratio=true]{../../modello/monolith}
	}

	\lfoot{
		\DocTitle{} \\
		% differenzia a seconda che \DocVersion{} stampi testo o no
		\setbox0=\hbox{\DocVersion{}\unskip}\ifdim\wd0=0pt
			% nulla
		\else
			v \DocVersion{}
		\fi
	}
	\rfoot{Pagina \thepage{} di \pageref{LastPage}}

	% Visualizza una linea orizzontale in cima e in fondo alla pagina
	\renewcommand{\headrulewidth}{0.3pt}
	\renewcommand{\footrulewidth}{0.3pt}
}
\setlength{\headheight}{30pt}
\pagestyle{plain}

% allarga l'header a tutta la pagina
%\fancyhfoffset[L]{\oddsidemargin + \hoffset + 1in}
%\fancyhfoffset[R]{\evensidemargin + \marginparwidth - \marginparsep}

% Per inserire del codice sorgente formattato
\usepackage{listings}

\lstset{
  extendedchars=true,          % lets you use non-ASCII characters
  inputencoding=utf8,   % converte i caratteri utf8 in latin1, richiede \usepackage{listingsutf8} anzichè listings
  basicstyle=\ttfamily,        % the size of the fonts that are used for the code
  breakatwhitespace=false,     % sets if automatic breaks should only happen at whitespace
  breaklines=true,             % sets automatic line breaking
  captionpos=t,                % sets the caption-position to top
  commentstyle=\color{mygreen},   % comment style
  frame=none,               % adds a frame around the code
  keepspaces=true,            % keeps spaces in text, useful for keeping indentation of code (possibly needs columns=flexible)
  keywordstyle=\bfseries,     % keyword style
  numbers=none,               % where to put the line-numbers; possible values are (none, left, right)
  numbersep=5pt,              % how far the line-numbers are from the code
  numberstyle=\color{mygray}, % the style that is used for the line-numbers
  rulecolor=\color{black},    % if not set, the frame-color may be changed on line-breaks within not-black text (e.g. comments (green here))
  showspaces=false,           % show spaces everywhere adding particular underscores; it overrides 'showstringspaces'
  showstringspaces=false,     % underline spaces within strings only
  showtabs=false,             % show tabs within strings adding particular underscores
  stepnumber=5,               % the step between two line-numbers. If it's 1, each line will be numbered
  stringstyle=\color{red},    % string literal style
  tabsize=4,                  % sets default tabsize
  firstnumber=1      % visualizza i numeri dalla prima linea
}

% Permetti di utilizzare il grassetto per i caratteri Typewriter (per es. il font di \code{...} e \file{...})
\usepackage[T1]{fontenc}
\usepackage{lmodern}

% Per poter usare il carattere ° (anche nelle formule matematiche)
\usepackage{gensymb}
\usepackage{newunicodechar}
\newunicodechar{°}{\degree}

% Per importare svg
\usepackage{svg}
%\setsvg{inkscape = inkscape -z -D}

% Per numerare le tabelle e le figure con la sezione in cui si trovano 
\usepackage{amsmath}
\numberwithin{figure}{section}
\numberwithin{table}{section}
