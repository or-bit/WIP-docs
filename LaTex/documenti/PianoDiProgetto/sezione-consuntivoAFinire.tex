\section{Consuntivo parziale}
In questa sezione vengono riportate le spese effettivamente sostenute. Verrà mostrato l'effettivo consumo di ore sia per ruolo che per persona ed in base al risultato avremo un bilancio:
\begin{itemize}
	\item positivo: se il consuntivo è inferiore al preventivo;
	\item negativo: se il consuntivo è maggiore al preventivo;
	\item in pari: se il consuntivo rispecchia a pieno il preventivo.
\end{itemize}
I valori positivi indicano un eccesso di ore, i negativi un consumo inferiore a quello preventivato.

\subsection{Analisi dei Requisiti}
La tabella sottostante riporta la differenza tra preventivo e consuntivo del periodo di \AR{} divisa per ruolo.
\begin{table}[H]
	\centering
	\begin{tabular}{|c|c|c|}
		\hline
		\textbf{Ruolo} &
		\textbf{Ore} &
		\textbf{Costo} \\
		\hline
		\Responsabile & -1 & -30\\
		\hline
		\Amministratore & 0 & 0\\
		\hline
		\Analista & 2 & 50\\
		\hline
		\Progettista & 0 & 0 \\
		\hline
		\Verificatore & 0 & 0\\
		\hline
		\Programmatore & 0 & 0 \\
		\hline
		\textbf{Totale} & \textbf{1} & \textbf{20} \\
		\hline
	\end{tabular}
	\caption{Differenza preventivo consuntivo per ruolo, periodo di \AR}
\end{table}

La tabella sottostante riporta la differenza tra preventivo e consuntivo della periodo di \AR{} divisa per componente.
\begin{table}[H]
	\centering
	\begin{tabular}{|l|c|c|c|c|c|c|c|}
		\hline
		\textbf{Nominativo} & 
		\multicolumn{6}{c|}{\textbf{Ore per ruolo}} & 
		\textbf{Ore totali} \\
		& Re & Am & An & Pj & Pr & Ve & \\
		\hline
		Nicola Dal Maso & & &1 & & & & 1 \\
		Lorenzo Ferrarin & & &1.5 & & & & 1.5 \\
		Beatrice Guerra & -2 & & & & & & -2 \\
		Marco Ponchia & & & & & & & 0 \\
		Tommaso Rosso & & & & & & & 0 \\
		Alice V. Sasso & & & & & & & 0 \\
		Mattia Zecchinato & 1& &-0.5 & & & & .5  \\
		\hline
	\end{tabular}
	\caption{Differenza preventivo consuntivo per componente, periodo di \AR}
\end{table}
\subsubsection{Conclusioni}
Per completare il periodo di \AR{} è stata necessaria un'ora di lavoro in più di quanto preventivato, con un aumento di spesa di \textbf{20€}.
\subsubsection{Impatto sul preventivo a finire}
Non essendo il periodo di \AR{} inclusa nella proposta, non ci saranno ripercussioni nel preventivo. Per quanto riguarda il preventivo totale, comprendente le ore non rendicontate, lo scostamento rilevato non è influente poiché corrisponde a meno di un'ora di lavoro.

\subsection{Analisi di Dettaglio}
La tabella sottostante riporta la differenza tra preventivo e consuntivo del periodo di \AD{} divisa per ruolo.
\begin{table}[H]
	\centering
	\begin{tabular}{|c|c|c|}
		\hline
		\textbf{Ruolo} &
		\textbf{Ore} &
		\textbf{Costo} \\
		\hline
		\Responsabile & 0 & 0\\
		\hline
		\Amministratore & 0 & 0\\
		\hline
		\Analista & 5 & 125\\
		\hline
		\Progettista & 0 & 0 \\
		\hline
		\Verificatore & 0 & 0\\
		\hline
		\Programmatore & 0 & 0 \\
		\hline
		\textbf{Totale} & \textbf{5} & \textbf{125} \\
		\hline
	\end{tabular}
	\caption{Differenza preventivo consuntivo per ruolo, periodo di \AD}
\end{table}

La tabella sottostante riporta la differenza tra preventivo e consuntivo della periodo di \AD{} divisa per componente.
\begin{table}[H]
	\centering
	\begin{tabular}{|l|c|c|c|c|c|c|c|}
		\hline
		\textbf{Nominativo} & 
		\multicolumn{6}{c|}{\textbf{Ore per ruolo}} & 
		\textbf{Ore totali} \\
		& Re & Am & An & Pj & Pr & Ve & \\
		\hline
		Nicola Dal Maso & & & & & & & 0 \\
		Lorenzo Ferrarin & & & & & & & 0 \\
		Beatrice Guerra & & &2 & & & & 2 \\
		Marco Ponchia & & & & & & & 0 \\
		Tommaso Rosso & & & & & & & 0 \\
		Alice V. Sasso & & & 2& & & & 2 \\
		Mattia Zecchinato & & &1 & & & & 1  \\
		\hline
	\end{tabular}
	\caption{Differenza preventivo consuntivo per componente, periodo di \AD}
\end{table}
\subsubsection{Conclusioni}
Per completare il periodo di \AD{} è stato necessario un investimento non preventivato di cinque ore, che ha portato ad un aumento dei costi di \textbf{125€}. Lo sforamento dal preventivo è stato causato da una visione ottimistica dei tempi e allo studio di alcuni punti risultati carenti dopo il periodo di \AR{}, sono stati quindi analizzati i periodi successivi e risultano realistici per la realizzazione del progetto.
\subsubsection{Impatto sul preventivo a finire}
Questa fase vista la sua durata ha avuto un impatto relativamente importante sul preventivo a finire, a causa delle modifiche che si sono rese necessarie.
È stata consumata una buona parte del margine previsto sul preventivo iniziale, che ci obbligherà a organizzare gli impegni per ruolo dei membri del gruppo in maniera più efficiente per le fasi successive.

\subsection{Progettazione Architetturale}
La tabella sottostante riporta la differenza tra preventivo e consuntivo del periodo di \PA{} divisa per ruolo.
\begin{table}[H]
	\centering
	\begin{tabular}{|c|c|c|}
		\hline
		\textbf{Ruolo} &
		\textbf{Ore} &
		\textbf{Costo} \\
		\hline
		\Responsabile & 0 & 0\\
		\hline
		\Amministratore & 0 & 0\\
		\hline
		\Analista & -2 & -50\\
		\hline
		\Progettista & -2 & -44 \\
		\hline
		\Verificatore & 1 & 15 \\
		\hline
		\Programmatore & 0 & 0 \\
		\hline
		\textbf{Totale} & \textbf{-3} & \textbf{-79} \\
		\hline
	\end{tabular}
	\caption{Differenza preventivo consuntivo per ruolo, periodo di \PA}
\end{table}

La tabella sottostante riporta la differenza tra preventivo e consuntivo della periodo di \PA{} divisa per componente.
\begin{table}[H]
	\centering
	\begin{tabular}{|l|c|c|c|c|c|c|c|}
		\hline
		\textbf{Nominativo} & 
		\multicolumn{6}{c|}{\textbf{Ore per ruolo}} & 
		\textbf{Ore totali} \\
		& Re & Am & An & Pj & Pr & Ve & \\
		\hline
		Nicola Dal Maso & & & & & & & 0 \\
		Lorenzo Ferrarin & & & & & & & 0 \\
		Beatrice Guerra & & & -2& & & 1& -1 \\
		Marco Ponchia & & & & & & & 0 \\
		Tommaso Rosso & & & & & & & 0 \\
		Alice V. Sasso & & & 1& -2& & & -1 \\
		Mattia Zecchinato & & & -1& & & & -1 \\
		\hline
	\end{tabular}
	\caption{Differenza preventivo consuntivo per componente, periodo di \PA}
\end{table}
\subsubsection{Conclusioni}
Durante il periodo di \PA{} sono state necessarie delle piccole riorganizzazioni nell'impegno orario per alcuni componenti, che non hanno mai riguardato un periodo superiore alle due ore a persona.
I cambiamenti sono stati resi necessari dal verificarsi di uno dei rischi preventivati, la rottura di un componente hardware da parte di un membro del gruppo, che non ha quindi portato ulteriori disguidi venendo affrontato con tempestività.
Queste scelte hanno portato ad un risparmio di tre ore e di \textbf{-79€} sul preventivo.
\subsubsection{Impatto sul preventivo a finire}
Il risparmio ottenuto in questo periodo, sia dal punto di vista dei costi, che da quello dell'impegno orario, ha portato ad un riavvicinamento al preventivo iniziale. Il riassestamento delle ore per alcune persone ha permesso di sistemare gli squilibri creatisi nei periodi precedenti evitando così un superamento del limite massimo delle ore. Questo ci permetterà di mantenere le fasi successive così come preventivato, senza apportare ulteriori modifiche.

\subsection{\PD{} e \Cod{}}
All'inizio dei periodi di \PD{} e \Cod{} è stato necessario effettuare una ripianificazione delle ore, data la decisione da parte del gruppo di posticipare la Revisione di Qualifica. Grazie all'esperienza acquisita fin'ora è stato possibile pianificare in modo corretto le attività, incrementando di 15 ore l'impegno in termini di ore/persona complessivo. La tabella seguente riporta la suddivisione di tale impegno per i diversi ruoli:
\begin{table}[H]
	\centering
	\begin{tabular}{|c|c|c|}
		\hline
		\textbf{Ruolo} &
		\textbf{Ore} &
		\textbf{Costo} \\
		\hline
		\Responsabile & 0 & 0\\
		\hline
		\Amministratore & 0 & 0\\
		\hline
		\Analista & 0 & 0\\
		\hline
		\Progettista & 10 & 220 \\
		\hline
		\Verificatore & 2 & 30 \\
		\hline
		\Programmatore & 3 & 45 \\
		\hline
		\textbf{Totale} & \textbf{15} & \textbf{295} \\
		\hline
	\end{tabular}
	\caption{Differenza preventivo consuntivo per ruolo, periodo di \PA}
\end{table}

\subsubsection{Conclusioni}
La scelta di posticipare la partecipazione alla Revisione di Qualifica, ha comportato un incremento delle ore da rendicontare, ma ha permesso al gruppo di lavorare in maniera più ordinata e conforme alle norme ed ottenendo anche una maggiore consapevolezza degli strumenti utilizzati.

\subsubsection{Impatto sul preventivo a finire}
L'aumento delle ore ha comportato un diretto aumento dei costi, che portano il preventivo a finire ad allinearsi al preventivo iniziale presentato durante la gara d'appalto.
La situazione attuale comporterà un'attenzione particolare a possibili variazioni durante la fase di \VV{}.

\subsection{\VV{}}
All'inizio dei periodi di \VV{} è stato necessario effettuare una ripianificazione delle ore, data dalla necessità di incrementare il numero di ore dedicate alla programmazione, non per questo però limitando le ore dedicate alla \VV{}, attività principale per questa fase. La tabella seguente riporta la suddivisione di tale impegno per i diversi ruoli:
\begin{table}[H]
	\centering
	\begin{tabular}{|c|c|c|}
		\hline
		\textbf{Ruolo} &
		\textbf{Ore} &
		\textbf{Costo} \\
		\hline
		\Responsabile & -2 & -60\\
		\hline
		\Amministratore & -4 & -80\\
		\hline
		\Analista & 0 & 0\\
		\hline
		\Progettista & -4 & -88 \\
		\hline
		\Verificatore & 12 & 180 \\
		\hline
		\Programmatore & 12 & 180 \\
		\hline
		\textbf{Totale} & \textbf{14} & \textbf{146} \\
		\hline
	\end{tabular}
	\caption{Differenza preventivo consuntivo per ruolo, periodo di \VV}
\end{table}

La tabella sottostante riporta la differenza tra preventivo e consuntivo della periodo di \VV{} divisa per componente.
\begin{table}[H]
	\centering
	\begin{tabular}{|l|c|c|c|c|c|c|c|}
		\hline
		\textbf{Nominativo} & 
		\multicolumn{6}{c|}{\textbf{Ore per ruolo}} & 
		\textbf{Ore totali} \\
		& Re & Am & An & Pj & Pr & Ve & \\
		\hline
		Nicola Dal Maso & & & &-2 & 4& & 2 \\
		Lorenzo Ferrarin & & & & & & 2& 2 \\
		Beatrice Guerra & & & & &2 & & 2 \\
		Marco Ponchia & & & & & 2& & 2 \\
		Tommaso Rosso & & & & -4& 4&2 & 2 \\
		Alice V. Sasso & & -2& & & & 4& 2 \\
		Mattia Zecchinato &-2 & & & & &4 & 2 \\
		\hline
	\end{tabular}
	\caption{Differenza preventivo consuntivo per componente, periodo di \VV}
\end{table}

\subsubsection{Conclusioni}
In questa fase è stata necessario un riassestamento delle risorse in quanto dopo la Revisione di Qualifica, abbiamo dovuto adeguare il framework, causando un incremento delle ore, principalmente per il ruolo di Programmatore.

\subsubsection{Impatto sul preventivo a finire}
L'incremento delle ore dei Programmatori, nonostante l'adeguamento delle altre risorse ha comportato una variazione sul preventivo di \textbf{146€}, portando così il totale rendicontato a 13585€ e causando una costo di \textbf{85€} non previsto dal preventivo iniziale presentato durante la gara di appalto.