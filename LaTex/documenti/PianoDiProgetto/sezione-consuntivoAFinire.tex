\section{Consuntivo a finire}
In questa sezione vengono riportate le spese effettivamente sostenute. Verrà mostrato l'effettivo consumo di ore sia per ruolo che per persona ed in base al risultato avremo un bilancio:
\begin{itemize}
	\item positivo: se il consuntivo è inferiore al preventivo;
	\item negativo: se il consuntivo è maggiore al preventivo;
	\item in pari: se il consuntivo rispecchia a pieno il preventivo.
\end{itemize}
I valori positivi indicano un eccesso di ore, i negativi un consumo inferiore a quello preventivato.

\subsection{Analisi}
La tabella sottostante riporta la differenza tra preventivo e consuntivo della fase di \AR{} divisa per ruolo.
\begin{table}[H]
	\centering
	\begin{tabular}{|c|c|c|}
		\hline
		\textbf{Ruolo} &
		\textbf{Ore} &
		\textbf{Costo} \\
		\hline
		\Responsabile & -1 & -30\\
		\hline
		\Amministratore & 0 & 0\\
		\hline
		\Analista & 2 & 50\\
		\hline
		\Progettista & 0 & 0 \\
		\hline
		\Verificatore & 0 & 0\\
		\hline
		\Programmatore & 0 & 0 \\
		\hline
		\textbf{Totale} & \textbf{1} & \textbf{20} \\
		\hline
	\end{tabular}
	\caption{Differenza preventivo consuntivo per ruolo, fase di \AR}
\end{table}

La tabella sottostante riporta la differenza tra preventivo e consuntivo della fase di \AR{} divisa per componente.
\begin{table}[H]
	\centering
	\begin{tabular}{|l|c|c|c|c|c|c|c|}
		\hline
		\textbf{Nominativo} & 
		\multicolumn{6}{c|}{\textbf{Ore per ruolo}} & 
		\textbf{Ore totali} \\
		& Re & Am & An & Pj & Pr & Ve & \\
		\hline
		Nicola Dal Maso & & &1 & & & & 1 \\
		Lorenzo Ferrarin & & &1.5 & & & & 1.5 \\
		Beatrice Guerra & -2 & & & & & & -2 \\
		Marco Ponchia & & & & & & & 0 \\
		Tommaso Rosso & & & & & & & 0 \\
		Alice V. Sasso & & & & & & & 0 \\
		Mattia Zecchinato & 1& &-0.5 & & & & .5  \\
		\hline
	\end{tabular}
	\caption{Differenza preventivo consuntivo per componente, fase di \AR}
\end{table}
\subsubsection{Conclusioni}
Per completare la fase di \AR{} è stata necessaria un'ora di lavoro in più di quanto preventivato, con un aumento di spesa di \textbf{20€}. Non essendo la fase di \AR{} inclusa nella proposta, non ci saranno ripercussioni nel preventivo.

