\letteraGlossario{U}
\definizione{UML}
Acronimo per Unified Modeling Language, è un linguaggio visuale di modellazione e specifica basato sul paradigma object-oriented.

\definizione{UNIX}
UNIX è un sistema operativo portabile per computer inizialmente sviluppato da un gruppo di ricerca dei laboratori AT\&T e Bell Laboratories. \`{E} la base di partenza di \glossario{Linux}.\\
I sistemi Unix-like, detti anche di tipo Unix, compatibili con Unix o *nix, sono sistemi operativi aderenti in larga parte agli standard derivati da Unix, tra cui la Single UNIX Specification e POSIX.

\definizione{Utente cliente}
Il cliente è il tipo di utente dell’applicazione che sfrutta quest’ultima per consultare il menù, selezionare i piatti desiderati ed effettuare l’ordinazione al ristorante.

\definizione{Utente cuoco}
Il cuoco è il tipo di utente dell’applicazione che sfrutta quest’ultima per visualizzare le ordinazione da preparare e notificare quando sono pronte.

\definizione{Utente direttore}
Il direttore è il tipo di utente dell’applicazione che sfrutta quest’ultima per controllare e gestire il flusso di ordinazioni, il magazzino e le consegne.

\definizione{Utente fattorino}
Il fattorino è il tipo di utente dell’applicazione che sfrutta quest’ultima per visualizzare la lista di consegne da effettuare e confermare quando ne completa una.

\definizione{Utente responsabile degli acquisti}
Il responsabile degli acquisti è il tipo di utente dell’applicazione che sfrutta quest’ultima per visualizzare la lista degli acquisti e spuntare quelli effettuati.

\definizione{UTF-8}
UTF-8, ovvero Unicode Transformation Format, 8 bit, è una codifica dei caratteri Unicode in sequenze di lunghezza variabile di byte.
\clearpage