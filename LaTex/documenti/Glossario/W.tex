\letteraGlossario{W}
\definizione{Walkthrough}
Tecnica di analisi statica per il testo o il codice. Fa parte dei metodi di lettura. Ha come obiettivo rilevare la presenza di difetti ed eseguire una lettura critica del prodotto in esame a largo spettro e senza l’assunzione di presupposti. Viene svolta da gruppi di persone che, in caso di codice, simulano le possibili esecuzioni.

\definizione{WebStorm}
WebStorm è un \glossario{IDE} per \glossario{JavaScript} sviluppato da \glossario{JetBrains}.\\
\url{https://www.jetbrains.com/webstorm/}

\definizione{Widget}
Un widget è un \glossario{elemento grafico} dell'interfaccia utente. Attraverso questi elementi interattivi (ad esempio un \glossario{Bottone}) viene reso più intuitivo l'utilizzo dell'applicazione per l'utente.

\definizione{Windows}
Microsoft Windows è una famiglia di ambienti operativi e sistemi operativi dedicati ai personal computer, alle workstation, ai server e agli smartphone.\\
\url{https://www.microsoft.com/it-it/windows/view-all}

\definizione{Windows Phone}
Windows Phone (spesso abbreviato in WP) è una famiglia di sistemi operativi per smartphone di Microsoft. Con la distribuzione di \glossario{Windows} 10, il sistema operativo per smartphone e tablet ha cambiato nome, diventando \glossario{Windows} 10 Mobile.

\definizione{Workflow}
Schema che descrive l’automatizzazione delle procedure e i processi di lavoro collaborativo.

\definizione{W3C}
Sintesi di World Wide Web Consortium \`e la principale organizzazione per gli standard del World Wide Web.\\
\url{http://www.w3c.it/}

\definizione{W3C Recommendation}
Standard formalmente dichiarati da parte del \glossario{W3C}.
\clearpage
