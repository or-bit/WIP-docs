\letteraGlossario{E}
\definizione{ECMAScript}
Vedi \textit{JavaScript}.

\definizione{Elemento}
Elemento del \glossario{framework} generico che può essere un \glossario{elemento grafico}, un \glossario{elemento funzionale}, un \glossario{elemento di input}, un \glossario{elemento di output} oppure un insieme di questi.

\definizione{Elemento funzionale}
Elemento del \glossario{framework} con una determinata funzionalità che ha lo scopo realizzare le operazioni esposte solitamente all’utente attraverso l’interfaccia grafica(\glossario{elemento grafico}).

\definizione{Elemento grafico}
Elemento dell’interfaccia grafica del \glossario{framework} con lo scopo di poter visualizzare dei dati o permettere all’utente di interagire.

\definizione{Elemento di input}
\glossario{Elemento grafico} che può essere aggiunto alla \glossario{bubble generica} per svolgerne le funzionalit\`a di input. Ogni elemento di input verr\`a assegnato ad una variabile all’interno della \glossario{bubble memory}. Gli elementi di input possono essere di vari tipi. Per maggiori dettagli sui tipi di input vedere sezione xxxxx del documento yyyyyy.

\definizione{Elemento di output}
\glossario{Elemento grafico} che può essere aggiunto alla \glossario{bubble generica} per svolgerne le funzionalit\`a di output. Ogni elemento di output verr\`a assegnato ad una variabile all’interno della \glossario{bubble memory}. Gli elementi di output possono essere di vari tipi. Per maggiori dettagli sui tipi di output vedere sezione xxxxx del documento yyyyyy.

\definizione{ESLint}
Strumento di linting per \glossario{JavaScript}.\\
\url{http://eslint.org/}
\clearpage