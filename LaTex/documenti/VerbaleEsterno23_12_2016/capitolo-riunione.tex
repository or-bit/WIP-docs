\section{Riunione}
\subsection{Ordine del Giorno}

\begin{itemize}
	\item Documentazione e Manualistica
	\item Target per le bubble
	\item \glossario{API}
	\item Demo
	\item Desktop e Mobile
	\item Proposte e Suggerimenti
\end{itemize}

\subsection{Dialogo con RedBabel}

\subsubsection{Documentazione e Manualistica}
La documentazione non accademica, rivolta all’utente finale, come un manuale utente, se realizzata può essere prodotta in forma di documento non verboso o come pagina web esplicativi delle funzionalità del prodotto. 
Esse saranno incluse nel repository di progetto e dovranno essere prodotte in lingua inglese.

\subsubsection{Ruoli}
Gli utilizzatori delle Bubble possono avere diversi ruoli. Si può così compiere una distinzione fra chi gestisce la Bubble e ne determina il funzionamento, e gli utilizzatori.
Grazie a questa distinzione è possibile considerare Bubble per l’uso formale. 
L’amministratore si può così porre ad un livello di controllo più alto dando loro la possibilità di utilizzare le Bubble in contesti di rapporti lavorativi e anche di servizi e commercio.

\subsubsection{Idee}
I Ruoli e l’integrazione di servizi di terze parti sono considerate buone idee.
Si consiglia un dialogo con personale in funzione di amministrazione che abbiano rapporti con il pubblico, per esempio amministrazione universitaria, e di considerare i rapporti fra chi offre servizi e chi li riceve, per pensare a strumenti da realizzare per permettere che questi rapporti avvengano via applicazioni di messaggistica.  

\subsubsection{Target delle Bubble}
Il target sono gli utenti di RocketChat, non ci sono preferenze per da parte di RedBabel sul tipo di Bubble o sul target delle Bubble da produrre. 

\subsubsection{API esterne e APIKey}
Per quanto riguarda l’utilizzo di \glossario{API} esterne che richiedano autenticazione o key si rimanda l’onere all’amministratore della bolla.

\subsubsection{Configurazione}
La parte di configurazione della bolla va astratta dai concetti di input output logica e stato.

\subsubsection{Demo}
La dimostrazione delle potenzialità del framework è parte integrante del capitolato.
Va posto l’accento sulla direzione da prendere e non sui dettagli tecnici in modo da avere una visuale sulle Bubble da implementare e includere nella demo. Si consiglia di prendere in considerazione due/tre Bubble.
Nella dimostrazione deve essere presente la distinzione fra ruoli, mostrando quindi una situazione di interazione di utenti in parità di livello e una situazione di customer service, se presenti nelle bolle considerate. 

\subsubsection{Mobile}
Quando è stato offert il capitolato non è stata considerata la tecnologia con cui RocketChat è stato realizzato per la parte mobile, si è considerato che durante il processo di creazione è probabile siano state usate varie tecnologie fra le quali meteor cordova per la generazione automatica del codice per mobile, sotto queste ipotesi è stato considerato di realizzare la compatibilità delle Bubble sotto forma di webview oppure considerarew di sviluppare un SDK specifico per mobile a sceltra fra un solo sistema: iOS o Android.  
Emerge l’inutilità di Bubble specifiche per mobile in un contesto di utilizzo desktop.

\clearpage
