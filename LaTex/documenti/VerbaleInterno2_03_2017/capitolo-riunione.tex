\section{Riunione}
\subsection{Ordine del Giorno}
\begin{itemize}
	\item Resoconto situazione
	\item Resoconto disponibilità oraria
	\item Discussione comportamento fattorino
	\item Discussione parti bubble contenute nel framework 
	\item Immagini progetto Visual Paradigm
\end{itemize}

\subsection{Discussione e decisioni}

\subsubsection{Resoconto situazione}
La struttura e la stesura delle scelte progettuali sono state completate. La visione d'insieme rimane generica si ridiscute quindi la decisione di design pattern, ma si conferma la scelta per le tecnologie utilizzate. Sono presi in considerazione i test con i requisiti e le componenti per avere una visione d'insieme.

\subsubsection{Resoconto disponibilità oraria}
Il gruppo rimane attivo nonostante lezioni e impegni lavorativi rendano difficile la conclusione anticipata dei lavori.

\subsubsection{Discussione comportamento fattorino}
Per rendere la demo il più utile e affidabile possibile per una situazione di utilizzo reale si è deciso di fare in modo che il fattorino per le consegne dell'impresa utilizzatrice della Bubble \& Eat possa prenotare le consegne in modo da gestire il proprio tempo ed il proprio percorso più efficientemente e prevenire di impostare un fittizio stato di consegna nel caso in cui esso stia portando a termina una data consegna e voglia prenderne in carico altre. 

\subsubsection{Discussione parti bubble contenute nel framework }
Si è deciso di considerare le parte delle bubble già presenti nel framework come direttamente collegate al framework e non di esclusiva competenza di ciascuna bubble, con conseguenze nelle descrizioni e nei grafici relativi ad esse.

\subsubsection{Immagini progetto Visual Paradigm}
Si è deciso di gestire le immagini all'interno di un progetto unico di Visual Paradigm in modo da poter controllarne esportazione e modifica.

\clearpage
