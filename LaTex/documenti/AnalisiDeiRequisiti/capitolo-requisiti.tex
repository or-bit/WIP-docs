\section{Requisiti}
Vengono ora presentati i requisiti emersi durante l’analisi del capitolato e di ogni \glossario{Caso d'uso} e i requisiti discussi nelle riunioni interne e con i proponenti.
Si è deciso di inserire i requisiti in una tabella dei requisiti per permettere una consultazione agevole degli stessi.
La tabella dei requisiti presenta i requisiti fino al massimo livello di dettaglio insieme alle loro caratteristiche, in particolare ne specifica:
\begin{itemize}
	\item codice;
	\item categoria di appartenenza fra:
	\begin{itemize}
		\item Obbligatori per i requisiti irrinunciabili per un qualsiasi stakeholder;
		\item Desiderabili per i requisiti non strettamente necessari, ma che offrono un valore aggiunto riconoscibile
		\item Opzionali per i requisiti relativamente utili o contrattabili in seguito
	\end{itemize}
	\item una descrizione esaustiva del requisito;
	\item le fonti dal quale il requisito ha avuto origine, sia essa l’analisi diretta del capitolato oppure il dialogo con i Proponenti e/o in base alle necessità architetturali ed implementative del progetto individuate tramite casi d'uso;	
\end{itemize}

\subsection{Tabella dei requisiti per il Framework Monolith}

\subsubsection{Requisiti di Vincolo}

\begin{longtable}{|P{1.5cm}|P{3cm}|P{6cm}|P{2.5cm}|}
	\hline \textbf{Codice} & \textbf{Categoria} & \textbf{Descrizione} & \textbf{Fonti} \\
	\hline R1 & Obbligatorio & Il \glossario{Framework} e la demo sono sviluppati in \glossario{Javascript}. & Capitolato \\
	\hline R2 & Obbligatorio & Il \glossario{Framework} è sviluppato come pachetto per \glossario{RocketChat} & Capitolato \\
	\hline R3 & Obbligatorio & Viene sviluppata una bubble di tipo to-do list. & Capitolato \\
	\hline R4 & Obbligatorio & Viene sviluppata una bubble per la gestione di un esercizio commerciale di ristorazione che permetta la gestione da parte del ristorante delle attività da svolgere che comprendono consegne da effettuare,  piatti da preparare, aqcuisti di merce, e permetta ai clienti l’ordinazione dei pasti. & Capitolato \\
	\hline R5 & Obbligatorio & Il \glossario{Framework} offre metodi con la funzionalità di modificare la \glossario{bubble memory}, quindi per gestire lo stato della bubble & VerbaleEsterno23\_12\_2016 \linebreak \hyperref[UC1.06.1]{UC1.06.1}  \\
	\hline R54 & Obbligatorio & La demo è caricata in \glossario{Heroku}. & Capitolato \\
	\hline
\end{longtable}

\subsubsection{Requisiti Funzionali}

\begin{longtable}{|P{1.5cm}|P{3cm}|P{6cm}|P{2.5cm}|}
	\hline \textbf{Codice} & \textbf{Categoria} & \textbf{Descrizione} & \textbf{Fonti} \\
	\hline R6 & Obbligatorio & L’utilizzatore del \glossario{Framework} può istanziare una bubble generica. & \hyperref[UC1.35]{UC1.35} \\
	\hline R7 & Obbligatorio & L’utilizzatore del \glossario{Framework} può rendere visibile una bubble generica precedentemente istanziata. & \hyperref[UC1.36]{UC1.36} \\
	\hline R8 & Obbligatorio & Il \glossario{Framework} offre la funzionalità di interfacciarsi con un database \glossario{MongoDB} fornito dall’utente. & VerbaleEsterno23\_12\_2016 \linebreak \hyperref[UC1.00]{UC1.00} \linebreak \hyperref[UC1.01]{UC1.01} \hyperref[UC1.02]{UC1.02} \\
	\hline R9 & Obbligatorio & Il \glossario{Framework} offre funzionalità di settaggio di durata delle bubble. & \hyperref[UC1.13]{UC1.13} \\
	\hline R10 & Obbligatorio & Il \glossario{Framework} offre funzionalità di terminazione bubble. & \hyperref[UC1.19]{UC1.19} \\
	\hline R11 & Obbligatorio & Il \glossario{Framework} offre la possibilità di aggiungere rimuovere o modificare elementi dalla bubble. & \hyperref[UC1.03.1]{UC1.03.1} \hyperref[UC1.04]{UC1.04} \hyperref[UC1.05.1]{UC1.05.1} \\
	\hline R12 & Obbligatorio & Il \glossario{Framework} offre la possibilità di inserire \glossario{label} nella bubble & \hyperref[UC1.27]{UC1.27} \\
	\hline R13 & Obbligatorio & Il \glossario{Framework} fornisce le funzionalità per impostare manualmente la posizione degli elementi grafici all'interno di una bubble generica. & \hyperref[UC1.23]{UC1.23} \\
	\hline R14 & Obbligatorio & Il \glossario{Framework} fornisce le funzionalità per mostrare e nascondere la posizione degli elementi grafici all'interno di una bubble generica. & \hyperref[UC1.21]{UC1.21} \linebreak \hyperref[UC1.22]{UC1.22} \\
	\hline R15 & Obbligatorio & Il \glossario{Framework} mette a disposizione funzionalità di creazione di notifica per le bubble & \hyperref[UC1.18]{UC1.18} \\
	\hline R16 & Obbligatorio & Il \glossario{Framework} mette a disposizione funzionalità di visualizzazione per le notifiche per le bubble & \hyperref[UC1.17]{UC1.17} \\
	\hline R33 & Obbligatorio & L’utilizzatore del \glossario{Framework} può inserire all'interno della bubble un TextView & \hyperref[UC1.26]{UC1.26} \\
	\hline R34 & Obbligatorio & L’utilizzatore del \glossario{Framework} può inserire all'interno della bubble un TextView editabile & \hyperref[UC1.33]{UC1.33} \\
	\hline R35 & Opzionale & Il \glossario{Framework} mette a disposizione metodi per la lettura di file \glossario{JSON} e il loro controllo rispetto ad uno schema specificato. & \hyperref[UC1.12]{UC1.12} \\
	\hline R36 & Opzionale & Il \glossario{Framework} offre la possibilità di monitorare gli utenti di Rocket.Chat utilizzatori della bubble. & \hyperref[UC1.14]{UC1.14} \\
	\hline R37 & Opzionale & Funzionalità per consultare lo storico delle interazioni che un singolo utente ha avuto con la bubble. & \hyperref[UC1.15.1]{UC1.15.1} \\
	\hline R38 & Opzionale & Il \glossario{Framework} offre la possibilità di stabilire una funzionalità di richiesta di file in input per la bubble & \hyperref[UC1.34]{UC1.34} \\
	\hline R39 & Opzionale & Il \glossario{Framework} offre la possibilità di stabilire una funzionalità di output per la bubble & \hyperref[UC1.24.1]{UC1.24.1} \\
	\hline R40 & Opzionale & Il \glossario{Framework} offre la possibilità di stabilire una funzionalità di output pdf per la bubble & \hyperref[UC1.20.1]{UC1.20.1} \\	
	\hline R41 & Opzionale & Il \glossario{Framework} offre la possibilità di inserire bottoni quali radio buttons e checkbox nella bubble & \hyperref[UC1.30]{UC1.30} \linebreak \hyperref[UC1.31]{UC1.31} \linebreak \hyperref[UC1.32]{UC1.32} \\
	\hline R42 & Opzionale & Il \glossario{Framework} offre metodi per fare matching con espressioni regolari
	 & \hyperref[UC1.10]{UC1.10} \\
	\hline R43 & Opzionale & Il \glossario{Framework} offre la possibilità di stabilire una funzionalità di  esecuzione ad un orario prestabilito & \hyperref[UC1.16]{UC1.16} \\
	\hline R44 & Opzionale & Il \glossario{Framework} offre la possibilità di chiamata di metodi di \glossario{API} esterne per riceverne file \glossario{JSON} & \hyperref[UC1.07.1]{UC1.07.1} \\
	\hline R45 & Opzionale & Il \glossario{Framework} mette a disposizione metodi per l’inserimento di immagini nelle bubble
	 & \hyperref[UC1.25]{UC1.25} \\
	\hline R46 & Opzionale & Il \glossario{Framework} offre metodi per la creazione di grafici a torta ed istogrammi
	 & \hyperref[UC1.28]{UC1.28} \linebreak \hyperref[UC1.29]{UC1.29}  \\	 
	 \hline R47 & Opzionale & Possibilità'di impostare un numero massimo di interazioni per ogni singolo utente di Rocket.Chat con la bubble. & \hyperref[UC1.08]{UC1.08} \\
	 \hline R48 & Opzionale & Possibilità di inserire un numero massimo di interazioni totali con la bubble & \hyperref[UC1.09]{UC1.09} \\
	 \hline R49 & Opzionale & Il \glossario{Framework} offre la funzionalità di specificare una dimensione massima per i file di input & \hyperref[UC1.11]{UC1.11} \\
	\hline
\end{longtable}


\subsection{Tabella dei requisiti per la bubble To-do list}

\subsubsection{Requisiti Funzionali}

\begin{longtable}{|P{1.5cm}|P{3cm}|P{6cm}|P{2.5cm}|}
	\hline \textbf{Codice} & \textbf{Categoria} & \textbf{Descrizione} & \textbf{Fonti} \\
	\hline R17 & Obbligatorio & La bubble to-do list permette all’utente la creazione di liste & \hyperref[UC2.1]{UC2.1} \\
	\hline R18 & Obbligatorio & La bubble to-do list offre la funzionalità di inserimento di nuovi elementi nella lista & \hyperref[UC2.2]{UC2.2} \\
	\hline R19 & Obbligatorio & La bubble to-do list offre la funzionalità di segnare come completati elementi della lista & \hyperref[UC2.3]{UC2.3} \\
	\hline R20 & Obbligatorio & La to-do list permette di settare un  remind come notifica statica & \hyperref[UC2.4]{UC2.4} \\
	\hline
\end{longtable}

\subsection{Tabella dei requisiti per la bubble Ristorazione}

\subsubsection{Requisiti Funzionali}

\begin{longtable}{|P{1.5cm}|P{3cm}|P{6cm}|P{2.5cm}|}
	\hline \textbf{Codice} & \textbf{Categoria} & \textbf{Descrizione} & \textbf{Fonti} \\
	\hline R21 & Obbligatorio & La bubble per la ristorazione può interfacciarsi con un database. In questo modo è possibile salvare informazioni riguardati i clienti e i dati riguardanti l’impresa, come merci e attività da svolgere. Grazie a questa funzione sarà possibile l’inserimento, la modifica e la cancellazione dei dati in base ai metodi utilizzati nella bubble. & \hyperref[UC3.2.1]{UC3.2.1} \hyperref[UC3.4.3]{UC3.4.3} \hyperref[UC3.5.1]{UC3.5.1} \hyperref[UC3.6.2]{UC3.6.2} \hyperref[UC3.7.1]{UC3.7.1} \hyperref[UC3.8.2]{UC3.8.2} \hyperref[UC3.9.1]{UC3.9.1} \hyperref[UC3.10.2]{UC3.10.2} \hyperref[UC3.11.2]{UC3.11.2} \hyperref[UC3.12.1]{UC3.12.1} \hyperref[UC3.12.6]{UC3.12.6} \hyperref[UC3.13.1.1]{UC3.13.1.1} \hyperref[UC3.13.2.3]{UC3.13.2.3} \hyperref[UC3.13.3.5]{UC3.13.3.5} \hyperref[UC3.13.4.4]{UC3.13.4.4} \hyperref[UC3.14.1]{UC3.14.1} \hyperref[UC3.14.6]{UC3.14.6} \hyperref[UC3.15.1]{UC3.15.1} \hyperref[UC3.15.5]{UC3.15.5} \\
	\hline R22 & Obbligatorio & Per permettere ai clienti l’utilizzo della bubble per la ristorazione sono incluse funzionalità di registrazione con il fine di raccogliere i dati per la consegna dei prodotti e permettere le ordinazioni. & \hyperref[UC3.1]{UC3.1} \\
	\hline R23 & Obbligatorio & I clienti possono consultare il menu del ristorante, in questo modo potranno informarsi sui cibi e sui prezzi per poi effettuare le scelte per l’ordine. & \hyperref[UC3.2]{UC3.2} \\
	\hline R24 & Obbligatorio & I clienti possono selezionare cibi e relative quantità per effettuare un ordine. & \hyperref[UC3.3]{UC3.3} \\
	\hline R25 & Obbligatorio & L’utente \glossario{Cuoco} può visualizzare la lista di piatti da preparare. & \hyperref[UC3.5]{UC3.5} \\
	\hline R26 & Obbligatorio & L’utente \glossario{Cuoco} può spuntare i piatti che ha preparato dalla lista di piatti da preparare  & \hyperref[UC3.6]{UC3.6} \\
	\hline R27 & Obbligatorio & L’utente responsabile degli acquisti può visualizzare la lista degli acquisti da effettuare. & \hyperref[UC3.7]{UC3.7} \\
	\hline R28 & Obbligatorio & L’utente responsabile degli acquisti ha la capacità di spuntare i prodotti che ha acquistato dalla lista acquisti.
	 & \hyperref[UC3.8]{UC3.8} \\
	\hline R29 & Obbligatorio & L’utente \glossario{direttore} ha la possibilità di modificare aggiungere e rimuovere le voci del menu con relativi prezzi & \hyperref[UC3.13]{UC3.13} \\
	\hline R30 & Obbligatorio & L’utente \glossario{direttore} può visualizzare la lista dei piatti da preparare & \hyperref[UC3.12.2]{UC3.12.2} \\
	\hline R31 & Obbligatorio & L’utente \glossario{direttore} può visualizzare la lista degli acquisti da effettuare. & \hyperref[UC3.14.2]{UC3.14.2} \hyperref[UC3.15.2]{UC3.15.2} \\
	\hline R32 & Obbligatorio & L’utente \glossario{direttore} può eliminare voci dalla lista dei piatti da preparare. & \hyperref[UC3.12]{UC3.12} \\
	\hline R53 & Obbligatorio & L’utente \glossario{direttore} può modificare la lista degli acquisti da effettuare aggiungendo ingredienti da aqcuistare oppure cancellandoli.
	 & \hyperref[UC3.14]{UC3.14} \linebreak \hyperref[UC3.15]{UC3.15} \\	 
	\hline R50 & Opzionale & L’utente \glossario{fattorino} può visualizzare la lista delle consegne da effettuare & \hyperref[UC3.9]{UC3.9} \\
	\hline R51 & Opzionale & L’utente \glossario{fattorino} ha la capacità di selezionare la consegna che intende effettuare & \hyperref[UC3.10]{UC3.10} \\
	\hline R52 & Opzionale & L’utente \glossario{fattorino} ha la capacità di confermare la consegna che ha effettuato & \hyperref[UC3.11]{UC3.11} \\
	\hline
\end{longtable}

\subsection{Tracciamento Fonti-Requisiti}

\begin{longtable}{|P{6cm}|P{6cm}|}
	\hline \textbf{Fonte} & \textbf{Requisiti}\\
	\hline Capitolato & R1 \linebreak R2 \linebreak R3 \linebreak R4 \linebreak R54 \\
	\hline VerbaleEsterno23\_12\_2016 & R5 \linebreak R8 \\
	\hline \hyperref[UC1.00]{UC1.00} & R8 \\
	\hline \hyperref[UC1.01]{UC1.01} & R8 \\
	\hline \hyperref[UC1.02]{UC1.02} & R8 \\
	\hline \hyperref[UC1.03.1]{UC1.03.1} & R11 \\
	\hline \hyperref[UC1.04]{UC1.04} & R11 \\
	\hline \hyperref[UC1.05.1]{UC1.05.1} & R11 \\
	\hline \hyperref[UC1.06.1]{UC1.06.1} & R5 \\
	\hline \hyperref[UC1.07.1]{UC1.07.1} & R44 \\
	\hline \hyperref[UC1.08]{UC1.08} & R47 \\
	\hline \hyperref[UC1.09]{UC1.09} & R48 \\
	\hline \hyperref[UC1.10]{UC1.10} & R42 \\
	\hline \hyperref[UC1.11]{UC1.11} & R49 \\	
	\hline \hyperref[UC1.12]{UC1.12} & R35 \\
	\hline \hyperref[UC1.13]{UC1.13} & R9 \\	
	\hline \hyperref[UC1.14]{UC1.14} & R36 \\
	\hline \hyperref[UC1.15.1]{UC1.15.1} & R37 \\
	\hline \hyperref[UC1.16]{UC1.16} & R43 \\
	\hline \hyperref[UC1.17]{UC1.17} & R16 \\
	\hline \hyperref[UC1.18]{UC1.18} & R15 \\
	\hline \hyperref[UC1.19]{UC1.19} & R10 \\
	\hline \hyperref[UC1.20.1]{UC1.20.1} & R40 \\
	\hline \hyperref[UC1.21]{UC1.21} & R14 \\
	\hline \hyperref[UC1.22]{UC1.22} & R14 \\
	\hline \hyperref[UC1.23]{UC1.23} & R13 \\
	\hline \hyperref[UC1.24.1]{UC1.24.1} & R39 \\
	\hline \hyperref[UC1.25]{UC1.25} & R45 \\
	\hline \hyperref[UC1.26]{UC1.26} & R33 \\
	\hline \hyperref[UC1.27]{UC1.27} & R12 \\
	\hline \hyperref[UC1.28]{UC1.28} & R46 \\
	\hline \hyperref[UC1.29]{UC1.29} & R46 \\
	\hline \hyperref[UC1.30]{UC1.30} & R41 \\
	\hline \hyperref[UC1.31]{UC1.31} & R41 \\
	\hline \hyperref[UC1.32]{UC1.32} & R41 \\
	\hline \hyperref[UC1.33]{UC1.33} & R34 \\
	\hline \hyperref[UC1.34]{UC1.34} & R38 \\
	\hline \hyperref[UC1.35]{UC1.35} & R6 \\
	\hline \hyperref[UC1.36]{UC1.36} & R7 \\	
	\hline \hyperref[UC2.1]{UC2.1} & R17 \\
	\hline \hyperref[UC2.2]{UC2.2} & R18 \\
	\hline \hyperref[UC2.3]{UC2.3} & R19 \\
	\hline \hyperref[UC2.4]{UC2.4} & R20 \\	
	\hline \hyperref[UC3.1]{UC3.1} & R22 \\
	\hline \hyperref[UC3.2]{UC3.2} & R23 \\
	\hline \hyperref[UC3.2.1]{UC3.2.1} & R21 \\
	\hline \hyperref[UC3.3]{UC3.3} & R24 \\
	\hline \hyperref[UC3.4.3]{UC3.4.3} & R21 \\
	\hline \hyperref[UC3.5]{UC3.5} & R25 \\
	\hline \hyperref[UC3.5.1]{UC3.5.1} & R21 \\
	\hline \hyperref[UC3.6]{UC3.6} & R26 \\
	\hline \hyperref[UC3.6.2]{UC3.6.2} & R21 \\
	\hline \hyperref[UC3.7]{UC3.7} & R27 \\
	\hline \hyperref[UC3.7.1]{UC3.7.1} & R21 \\
	\hline \hyperref[UC3.8]{UC3.8} & R28 \\
	\hline \hyperref[UC3.8.2]{UC3.8.2} & R21 \\
	\hline \hyperref[UC3.9]{UC3.9} & R50 \\
	\hline \hyperref[UC3.9.1]{UC3.9.1} & R21 \\
	\hline \hyperref[UC3.10]{UC3.10} & R51 \\
	\hline \hyperref[UC3.10.2]{UC3.3.10.2} & R21 \\
	\hline \hyperref[UC3.11]{UC3.11} & R52 \\
	\hline \hyperref[UC3.11.2]{UC3.11.2} & R21 \\
	\hline \hyperref[UC3.12]{UC3.12} & R32 \\
	\hline \hyperref[UC3.12.1]{UC3.12.1} & R21 \\
	\hline \hyperref[UC3.12.2]{UC3.12.2} & R30 \\
	\hline \hyperref[UC3.12.6]{UC3.12.6} & R21 \\
	\hline \hyperref[UC3.13]{UC3.13} & R29 \\
	\hline \hyperref[UC3.13.1.1]{UC3.13.1.1} & R21 \\
	\hline \hyperref[UC3.13.2.3]{UC3.13.2.3} & R21 \\
	\hline \hyperref[UC3.13.3.5]{UC3.13.3.5} & R21 \\
	\hline \hyperref[UC3.13.4.4]{UC3.13.4.4} & R21 \\
	\hline \hyperref[UC3.14]{UC3.14} & R53 \\
	\hline \hyperref[UC3.14.1]{UC3.14.1} & R21 \\
	\hline \hyperref[UC3.14.2]{UC3.14.2} & R31 \\
	\hline \hyperref[UC3.14.6]{UC3.14.6} & R21 \\
	\hline \hyperref[UC3.15]{UC3.15} & R53 \\
	\hline \hyperref[UC3.15.1]{UC3.15.1} & R21 \\
	\hline \hyperref[UC3.15.2]{UC3.15.2} & R31 \\
	\hline \hyperref[UC3.15.5]{UC3.15.5} & R21 \\
	\hline
\end{longtable}

\subsection{Tracciamento Requisiti-Fonti}

\begin{longtable}{|P{6cm}|P{6cm}|}
	\hline \textbf{Codice Requisito} & \textbf{Fonti}\\
	\hline R1 & Capitolato \\
	\hline R2 & Capitolato \\
	\hline R3 & Capitolato \\
	\hline R4 & Capitolato \\
	\hline R54 & Capitolato \\
	\hline R5 & VerbaleEsterno23\_12\_2016 \linebreak \hyperref[UC1.06.1]{UC1.06.1}  \\	
	\hline R6 & \hyperref[UC1.35]{UC1.35} \\
	\hline R7 & \hyperref[UC1.36]{UC1.36} \\
	\hline R8 & VerbaleEsterno23\_12\_2016 \linebreak \hyperref[UC1.00]{UC1.00} \linebreak \hyperref[UC1.01]{UC1.01} \linebreak \hyperref[UC1.02]{UC1.02} \\
	\hline R9 & \hyperref[UC1.13]{UC1.13} \\
	\hline R10 & \hyperref[UC1.19]{UC1.19} \\
	\hline R11 & \hyperref[UC1.03.1]{UC1.03.1} \linebreak \hyperref[UC1.04]{UC1.04} \linebreak \hyperref[UC1.05.1]{UC1.05.1} \\
	\hline R12 & \hyperref[UC1.27]{UC1.27} \\
	\hline R13 & \hyperref[UC1.23]{UC1.23} \\
	\hline R14 & \hyperref[UC1.21]{UC1.21} \linebreak \hyperref[UC1.22]{UC1.22} \\
	\hline R15 & \hyperref[UC1.18]{UC1.18} \\
	\hline R16 & \hyperref[UC1.17]{UC1.17} \\
	\hline R33 & \hyperref[UC1.26]{UC1.26} \\
	\hline R34 & \hyperref[UC1.33]{UC1.33} \\
	\hline R35 & \hyperref[UC1.12]{UC1.12} \\
	\hline R36 & \hyperref[UC1.14]{UC1.14} \\
	\hline R37 & \hyperref[UC1.15.1]{UC1.15.1} \\
	\hline R38 & \hyperref[UC1.34]{UC1.34} \\
	\hline R39 & \hyperref[UC1.24.1]{UC1.24.1} \\
	\hline R40 & \hyperref[UC1.20.1]{UC1.20.1} \\	
	\hline R41 & \hyperref[UC1.30]{UC1.30} \linebreak \hyperref[UC1.31]{UC1.31} \linebreak \hyperref[UC1.32]{UC1.32} \\
	\hline R42 & \hyperref[UC1.10]{UC1.10} \\
	\hline R43 & \hyperref[UC1.16]{UC1.16} \\
	\hline R44 & \hyperref[UC1.07.1]{UC1.07.1} \\
	\hline R45 & \hyperref[UC1.25]{UC1.25} \\
	\hline R46 & \hyperref[UC1.28]{UC1.28} \linebreak \hyperref[UC1.29]{UC1.29}  \\	 
	\hline R47 & \hyperref[UC1.08]{UC1.08} \\
	\hline R48 & \hyperref[UC1.09]{UC1.09} \\
	\hline R49 & \hyperref[UC1.11]{UC1.11} \\	
	\hline R17 & \hyperref[UC2.1]{UC2.1} \\
	\hline R18 & \hyperref[UC2.2]{UC2.2} \\
	\hline R19 & \hyperref[UC2.3]{UC2.3} \\
	\hline R20 & \hyperref[UC2.4]{UC2.4} \\	
	\hline R21 & \hyperref[UC3.2.1]{UC3.2.1} \linebreak \hyperref[UC3.4.3]{UC3.4.3} \linebreak \hyperref[UC3.5.1]{UC3.5.1} \linebreak \hyperref[UC3.6.2]{UC3.6.2} \linebreak \hyperref[UC3.7.1]{UC3.7.1} \linebreak \hyperref[UC3.8.2]{UC3.8.2} \linebreak \hyperref[UC3.9.1]{UC3.9.1} \linebreak \hyperref[UC3.10.2]{UC3.10.2} \linebreak \hyperref[UC3.11.2]{UC3.11.2} \linebreak  \hyperref[UC3.12.1]{UC3.12.1} \linebreak \hyperref[UC3.12.6]{UC3.12.6} \linebreak  \hyperref[UC3.13.1.1]{UC3.13.1.1} \linebreak \hyperref[UC3.13.2.3]{UC3.13.2.3} \linebreak \hyperref[UC3.13.3.5]{UC3.13.3.5} \linebreak \hyperref[UC3.13.4.4]{UC3.13.4.4} \linebreak \hyperref[UC3.14.1]{UC3.14.1} \linebreak \hyperref[UC3.14.6]{UC3.14.6} \linebreak \hyperref[UC3.15.1]{UC3.15.1} \linebreak \hyperref[UC3.15.5]{UC3.15.5} \\
	\hline R22 & \hyperref[UC3.1]{UC3.1} \\
	\hline R23 & \hyperref[UC3.2]{UC3.2} \\
	\hline R24 & \hyperref[UC3.3]{UC3.3} \\
	\hline R25 & \hyperref[UC3.5]{UC3.5} \\
	\hline R26 & \hyperref[UC3.6]{UC3.6} \\
	\hline R27 & \hyperref[UC3.7]{UC3.7} \\
	\hline R28 & \hyperref[UC3.8]{UC3.8} \\
	\hline R29 & \hyperref[UC3.13]{UC3.13} \\
	\hline R30 & \hyperref[UC3.12.2]{UC3.12.2} \\
	\hline R31 & \hyperref[UC3.14.2]{UC3.14.2} \linebreak \hyperref[UC3.15.2]{UC3.15.2} \\
	\hline R32 & \hyperref[UC3.12]{UC3.12} \\
	\hline R53 & \hyperref[UC3.14]{UC3.14} \linebreak \hyperref[UC3.15]{UC3.15} \\
	\hline R50 & \hyperref[UC3.9]{UC3.9} \\
	\hline R51 & \hyperref[UC3.10]{UC3.10} \\
	\hline R52 & \hyperref[UC3.11]{UC3.11} \\	
	\hline
\end{longtable}

\subsection{Riepilogo}

\begin{longtable}{|P{3cm}|P{3cm}|P{3cm}|P{3cm}|}
	\hline \textbf{Categoria} & \textbf{Obbligatorio} & \textbf{Opzionale} & \textbf{Desiderabile} \\
	\hline Funzionale & 30 & 18 & 0 \\
	\hline Prestazionale & 0 & 0 & 0 \\
	\hline Qualitativo & 0 & 0 & 0 \\
	\hline di Vincolo & 6 & 0 & 0 \\
	\hline
\end{longtable}
