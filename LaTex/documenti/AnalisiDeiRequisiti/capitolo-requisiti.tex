\section{capitolo-requisiti}
Vengono ora presentati i requisiti emersi durante l’analisi del capitolato, dei casi d’uso e discussi nelle riunioni interne e con i proponenti.
Si è deciso di inserire i requisiti in una tabella dei requisiti per permettere una consultazione agevole degli stessi.
La tabella dei requisiti presenta i requisiti fino al massimo livello di dettaglio insieme alle loro caratteristiche, in particolare ne specifica:
\begin{itemize}
	\item codice;
	\item categoria di appartenenza fra:
	\begin{itemize}
		\item Obbligatori per i requisiti irrinunciabili per un qualsiasi stakeholder;
		\item Desiderabili per i requisiti non strettamente necessari, ma che offrono un valore aggiunto riconoscibile
		\item Opzionali per i requisiti relativamente utili o contrattabili in seguito
	\end{itemize}
	\item una descrizione esaustiva del requisito;
	\item le fonti dal quale il requisito ha avuto origine, sia essa l’analisi diretta del capitolato oppure il dialogo con i Proponenti e/o in base alle necessità architetturali ed implementative del progetto individuate tramite casi d'uso;	
\end{itemize}

\subsection{Tabella dei requisiti}


\begin{longtable}{|P{1.5cm}|P{3cm}|P{6cm}|P{2.5cm}|}
	\hline \textbf{Codice} & \textbf{Categoria} & \textbf{Descrizione} & \textbf{Fonti} \\
	\hline 0.0.0 & Obbligatorio & Il framework è sviluppato in javascript. & Capitolato \\
\end{longtable}
