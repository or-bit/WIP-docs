\section{Introduzione}

\subsection{Scopo del documento}
Lo scopo di questo documento � quello di definire i requisiti emersi dall?analisi del capitolato 5.
Il presente, tra le altre cose, tratter� di:
\begin{itemize}
	\item Descrizione dei requisiti
	\item Descrizione dei casi d'uso
	\item Descrizione degli attori coinvolti
\end{itemize}

\subsection{Capitolato scelto}
Capitolato: C5 - \ProjectName{}: An interactive bubble provider \\
Proponente: \Proponente{} \\
Committente: \Committente{} \\

\subsection{Scopo del Prodotto}
\ScopoDelProdotto

\subsection{Glossario}
Insieme al progetto il gruppo fornir� anche un glossario nel quale saranno raccolti i termini tecnici, acronimi e le parole che necessitano di chiarimenti, che verranno utilizzate nella documentazione.
I vocaboli presenti nel documento \Glossario saranno marcati con una g al pedice.

\subsection{Riferimenti}
\subsubsection{Normativi}
\begin{itemize}
	\item \textbf{\NormeDiProgetto};
	\item\textbf{ Capitolato d'appalto C5:} \ProjectName{}: An interactive bubble provider: \url{http://www.math.unipd.it/~tullio/IS-1/2016/Progetto/C5.pdf};
	\item \textbf{Vincoli sull'organigramma del gruppo e sull'offerta tecnico-economica:} \\ \url{http://www.math.unipd.it/~tullio/IS-1/2016/Progetto/PD01b.html}.
\end{itemize}

\subsubsection{Informativi}
\begin{itemize}
	\item \textbf{Slide dell'insegnamento Ingegneria del Software modulo A:}
	\begin{itemize}
		\item Ciclo di vita del Software;
		\item Gestione di Progetto.
	\end{itemize}
	\url{http://www.math.unipd.it/~tullio/IS-1/2016/}
	\item \textbf{\textit{Software Engineering} - Ian Sommerville - 9th Edition (2011):}
	\begin{itemize}
		\item Part 4: Software Management.
	\end{itemize} 
\end{itemize}