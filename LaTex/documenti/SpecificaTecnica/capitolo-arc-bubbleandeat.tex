\subsection{Bubble \& Eat}

\subsubsection{Package}
Il package della demo Bubble \& eat è composto da due ulteriori packages: customer e ristorante. 
Il primo si riferisce soltato all'utente customer. Il secondo invece contiene i package di tutti gli altri attori:
\begin{itemize}
	\item manager;
	\item chef;
	\item delivery man;
	\item purchasing manager.
\end{itemize}
Nel package Ristorante, inoltre, è anche contenuto il pacchetto Order Gateway.\\
Lo scopo del package Order Gateway è quello di creare, modificare ed eliminare le ordinazioni interagendo con il database che le contiene.
Tutti gli attori interni ed esterni al ristorante hanno dipendenze verso l’Order Gateway.
%\begin{figure}[H]
%	\centering
%	\includegraphics[width=15cm]{../../documenti/SpecificaTecnica/diagrammi_img/framework.png}
%	\caption{Demo Bubble \& Eat}
%\end{figure}

\subsubsection{Package Order Gateway}
Il package Order Gateway si occupa di gestire le interazioni tra attori e database e le eventuali comunicazioni tra gli stessi attori. Il package internamente è composto dalle classi Order Gateway e Menu e dal package Orders.\\
Il package Orders definisce le ordinazioni e la collezione di quest’ultime che verranno poi costruite seguendo il Factory method come design pattern dalla classe Menu.\\
Tutte le interazioni con gli elementi descritti sopra sono rese disponibili agli attori dalla classe Order Gateway.
%\begin{figure}[H]
%	\centering
%	\includegraphics[width=15cm]{../../documenti/SpecificaTecnica/diagrammi_img/framework.png}
%	\caption{Package Order Gateway}
%\end{figure}

\subsubsection{Struttura delle classi}
%\begin{figure}[H]
%	\centering
%	\includegraphics[width=15cm]{../../documenti/SpecificaTecnica/diagrammi_img/framework.png}
%	\caption{Diagramma di struttura delle classi}
%\end{figure}

\subsubsection{Descrizione classi}

\paragraph{Demo Bubble\&eat::customer::Bubble customer}\mbox{}\\
\textbf{Descrizione:}\\
La classe permette al customer di consultare il menu, registrare i propri dati personali e inviare le proprie ordinazioni.\\
\textbf{Utilizzo:}\\
Questa classe attraverso l’Order gateway riceve il menu da consultare per mezzo del quale può creare la propria ordinazione.\\

\paragraph{Demo Bubble\&eat::Ristorante::manager::Bubble manager}\mbox{}\\
\textbf{Descrizione:}\\
La classe permette al manager di gestire il menu, il magazzino e le ordinazioni.\\
\textbf{Utilizzo:}\\
Questa classe permette di gestire tutte le operazioni fornite da Order Gateway ed ha la possibilità di manipolare il magazzino e le ordinazioni.\\

\paragraph{Demo Bubble\&eat::Ristorante::chef::Bubble chef}\mbox{}\\
\textbf{Descrizione:}\\
La classe permette allo chef di consultare le ordinazioni e di impostare lo stato di una ordinazione in modo da renderla disponibile alla consegna.\\
\textbf{Utilizzo:}\\
Questa classe attraverso l’Order Gateway recupera e visualizza le ordinazioni e offre la possibilità allo chef di modificarne lo stato.\\

\paragraph{Demo Bubble\&eat::Ristorante::deliveryman::Bubble deliveryman}\mbox{}\\
\textbf{Descrizione:}\\
La classe permette al delivery man di consultare gli ordini pronti alla consegna, di ottenere i dati personali dei clienti relativi alle consegne selezionate e di eliminarle una volta completate.\\
\textbf{Utilizzo:}\\
Questa classe attraverso l’Order Gateway riceve la lista delle ordinazioni da consegnare, le visualizza e permette di selezionare le consegne ed eliminarle una volta selezionate.

\paragraph{Demo Bubble\&eat::Ristorante::Purchasing Manager::Bubble Purchasing Manager}\mbox{}\\
\textbf{Descrizione:}\\
La classe permette al Purchasing Manager di recuperare e visualizzare la lista della fornitura da acquistare oltre ad eliminare gli acquisti dalla lista.\\
\textbf{Utilizzo:}\\
Questa classe recupera dall’Order Gateway la lista degli acquisti, la visualizza e consente di eliminare gli acquisti presenti nella stessa.

\paragraph{Demo Bubble\&eat::Ristorante::Order Gateway::Order Gateway}\mbox{}\\
\textbf{Descrizione:}\\
La classe permette a tutti gli attori di interagire con il database attraverso apposite funzioni e si occupa di gestire le interazioni con il menù e l’insieme delle ordinazioni contenute nell’Order Container. La classe si occupa inoltre di inviare le notifiche agli attori quando viene richiesto o in maniera automatica ed aggiorna le risorse e le liste in tempo reale.\\
\textbf{Utilizzo:}\\
Questa classe offre diverse funzionalità agli attori per manipolare il database, le ordinazioni contenute nell’Order Container ed il menù, che permette a sua volta la creazione delle ordinazioni. Queste funzioni saranno disponibili o meno a seconda del tipo di attore.\\
La classe aggiorna le risorse in modo automatico scalando gli ingredienti dai valori impostati per ogni piatto nel menù. Ha inoltre delle funzioni di notifica che possono essere impostate automaticamente durante le operazioni o chiamate da alcuni attori.

\paragraph{Demo Bubble\&eat::Ristorante::Order Gateway::Menu}\mbox{}\\
\textbf{Descrizione:}\\
La classe permette la creazione delle ordinazioni ed attraverso l’Order Gateway fornisce diverse operazioni agli utenti:
\begin{itemize}
	\item customer: recupero del menu e creazione dell’ordinazione;
	\item manager: recupero e modifica del menu.
\end{itemize}
\textbf{Utilizzo:}\\
Questa classe implementa il design pattern del Factory method e permette la creazione delle ordinazioni, inoltre fornisce dei metodi per modificare ed ottenere il menu.

\paragraph{Demo Bubble\&eat::Ristorante::Order Gateway::Ordinazione::Order}\mbox{}\\ 
\textbf{Descrizione:}\\
Questa classe definisce il contenuto dell’ordinazione. Contiene gli elementi scelti dal menu e i dati forniti dall’utente per la consegna.\\
L’ordinazione è definita anche da uno stato e la classe offre i metodi per modificarlo.
\textbf{Utilizzo:}\\
Questa classe permette di inserire gli elementi del menu nell'ordinazione e di modificare lo stato della stessa. 

\paragraph{Demo Bubble\&eat::Ristorante::Order Gateway::Ordinazione::Order Container}\mbox{}\\
\textbf{Descrizione:}\\
La classe permette la gestione di una collezione di ordinazioni definiti dalla classe Order.\\
\textbf{Utilizzo:}\\
Questa classe offre la possibilità di aggiungere, rimuovere e restituire elementi da una collezione di ordinazioni.