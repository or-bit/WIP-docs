\subsection{Test di validazione}

%\subsubsection{Framework}
%
%\begin{longtable}{|P{5cm}|P{5cm}|P{5cm}|}
%	\hline \textbf{Codice} & \textbf{Descrizione} & \textbf{Requisito} \\
%	\endfirshead
%	
%	\hline \test{S} & Data una GUI con più elementi e un elemento funzionale, simulare un input utente e far inviare un segnale dalla GUI alla bubble, seguendo tutto il percorso, per controllare che ciò che la GUI ritorna sia corretto & \\
%	\hline \test{S} & Data una GUI con più elementi e un elemento funzionale, simulare un input utente e far inviare un segnale non corretto dalla GUI alla bubble, seguendo tutto il percorso, per controllare che la GUI ritorni un errore e non si aggiorni le informazioni sbagliate & \\
%	\hline \test{S} & Data una GUI con più elementi e un elemento funzionale, simulare un evento originato dall'elemento funzionale e verificare che questo abbia ripercussioni sulla bubble generica e sulla GUI & \\
%	\hline \test{S} & Data una GUI con un elemento e più elementi funzionali, simulare un input utente e far inviare un segnale dalla GUI alla bubble, tracciando tutto il percorso, per controllare che ciò che la GUI ritorna sia corretto & \\
%	\hline \test{S} & Data una GUI con un elemento e più elementi funzionali, simulare un input utente e far inviare un segnale non corretto dalla GUI alla bubble, seguendo tutto il percorso, per controllare che la GUI ritorni un errore e non si aggiorni le informazioni sbagliate & \\
%	\hline \test{S} & Data una GUI con un elemento e più elementi funzionali, simulare un evento originato dall'elemento funzionale e verificare che questo abbia ripercussioni sulla bubble generica e sulla GUI & \\
%	\hline \test{S} & Data una GUI con più elementi e più elementi funzionali, simulare un input utente e far inviare un segnale dalla GUI alla bubble, tracciando tutto il percorso, per controllare che ciò che la GUI ritorna sia corretto & \\
%	\hline \test{S} & Data una GUI con più elementi e più elementi funzionali, simulare un input utente e far inviare un segnale non corretto dalla GUI alla bubble, seguendo tutto il percorso, per controllare che la GUI ritorni un errore e non si aggiorni le informazioni sbagliate & \\
%	\hline \test{S} & Data una GUI con un elemento e più elementi funzionali, simulare un evento originato dall'elemento funzionale e verificare che questo abbia ripercussioni sulla bubble generica e sulla GUI & \\
%	\hline
%	\caption{Test di sistema per il framework}
%\end{longtable}

\subsubsection{Bubble To-do list}

\begin{longtable}{|c|P{9cm}|c|c|}
	\hline \multicolumn{1}{|l|}{\textbf{Codice}} &  \multicolumn{1}{l|}{\textbf{Descrizione}} & \multicolumn{1}{l|}{\textbf{Stato}} & \multicolumn{1}{l|}{\textbf{Requisito}} \\ 
	\endfirsthead
	\hline \test{V} & Verificare  &  & \ref{AdR-L17}\\
	\hline
	\caption{Test di validazione}
\end{longtable}

