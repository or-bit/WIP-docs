\section{To-do List Functionalities guide}
\begin{figure}[H]
	\centering
	\includegraphics[width=7cm]{../../documenti/UserManualDemo/demo_screens/todo.png}
	\caption{To-do List}
\end{figure}
\subsection{Adding an Item to the List}
Type the text of the entry and click ``Submit''. The new item will appear on the list.
\subsection{Mark Items as completed}
Select the entries and click on ``Complete Selected'' to mark them as completed. 
\subsection{Removing Items From the List}
Select the entries and click on ``Delete Selected'' to remove them from the list.

\section{\DemoName{} Functionalities guide}
\subsection{Customer}
\subsubsection{Consulting the Menu}
\begin{figure}[H]
	\centering
	\includegraphics[width=7cm]{../../documenti/UserManualDemo/demo_screens/client_main.png}
	\caption{Customer Main Page}
\end{figure}
\paragraph{Customer Main Page}
In the Customer Main Page it is possible to access the menu by clicking the button labeled as ``Menu''.
\begin{figure}[H]
	\centering
	\includegraphics[width=7cm]{../../documenti/UserManualDemo/demo_screens/client_menu.png}
	\caption{Customer Menu Page}
\end{figure}
\paragraph{Menu Page }
In the Menu Page is possible for the Customer to consult the Menu presenting all the dishes available to order in the restaurant.
By clickyng on the ``Back'' button the Customer will be redirected to the Customer main page.

\subsubsection{Placing an Order}
\begin{figure}[H]
	\centering
	\includegraphics[width=7cm]{../../documenti/UserManualDemo/demo_screens/client_main.png}
	\caption{Customer Main Page}
\end{figure}
\paragraph{Customer Main Page}
In the Customer Main Page it is possible to access the Order page by by clicking ``Make a new Order'' button.
\begin{figure}[H]
	\centering
	\includegraphics[width=7cm]{../../documenti/UserManualDemo/demo_screens/client_dishes.png}
	\caption{Customer Order Page}
\end{figure}
\paragraph{New Order Page}
New Order Page shows all the available dishes and let the Customer set quantities for its order, either by inserting manually or by increasing and decreasing quantities through the specific buttons marked by the plus and minus symbols. The page also shows the total amount due for the order.
By clicking on Reset Order quantities for all the dishes will be reset to zero.
To coninue the order procedure a Customer must click ``Insert Info''. 
It is possible for a Customer to go back to the main page by clicking ``Back''.

\begin{figure}[H]
	\centering
	\includegraphics[width=7cm]{../../documenti/UserManualDemo/demo_screens/client_info.png}
	\caption{Customer's Info Page}
\end{figure}
\paragraph{Insert Personal Data}
In this page it is possible for the Customer to insert a Name and an Address in the specific form in order to have its order to be delivered at the specified location. To confirm the order a Customer must click on 'confirm order'. 
It is possible for a Customer to go back to the order page by clicking 'back'.

\begin{figure}[H]
	\centering
	\includegraphics[width=7cm]{../../documenti/UserManualDemo/demo_screens/client_summary.png}
	\caption{Customer Summary Page}
\end{figure}
\paragraph{Summary or Waiting Page}
After an order is placed the user is redirected to the summary/waiting page where a summary of the order is shown and the Customer is advised to wait until the dishes are are ready.

\begin{figure}[H]
	\centering
	\includegraphics[width=7cm]{../../documenti/UserManualDemo/demo_screens/client_notification.png}
	\caption{Customer Notification Page}
\end{figure}
\paragraph{Notification}
When the ordination is marked as completed by the Chef, a notification is sent to warn the Customer that its meal is ready.

\subsection{Chef}
\subsubsection{Consulting the Orders}
In the Chef order page is possible to view all the orders and to mark them as completed.

\subsection{Manager}
\subsubsection{Accessing the Menu Operations}
\begin{figure}[H]
	\centering
	\includegraphics[width=7cm]{../../documenti/UserManualDemo/demo_screens/admin_main.png}
	\caption{Manager Main Page}
\end{figure}
\paragraph{Manager Main Page}
In the Manager main page it is possible to access the Menu Operations page by clicking the button labeled as ``Menu Operations''.

\begin{figure}[H]
	\centering
	\includegraphics[width=7cm]{../../documenti/UserManualDemo/demo_screens/admin_menu.png}
	\caption{Manager Menu Page}
\end{figure}
\paragraph{Menu Operations}
In the menu operations page is possible for the Manager to decide whether to add, edit or delete the dishes.
By clicking ``Delete'' on a certain dish, it will be deleted from the restairant menu Menu.
By clicking ``Edit'' on a certain dish, the Manager will be redirected to the Edit Dish page where will be possible to Edit the specific dish.
By clicking ``Add'' the Manager will be redirected to Add Dish Page.
It is possible for the manager to go back to the main page by clicking 'back'.

\begin{figure}[H]
	\centering
	\includegraphics[width=7cm]{../../documenti/UserManualDemo/demo_screens/admin_edit.png}
	\caption{Manager Edit dish Page}
\end{figure}
\paragraph{Edit Dish Page}
In the edit dish page is shown a form containing the current dish name and price that can be edited and saved by clicking ``Submit''.
It is possible for the Manager to go back to the Menu Operation Page by clicking 'back'.

\begin{figure}[H]
	\centering
	\includegraphics[width=7cm]{../../documenti/UserManualDemo/demo_screens/admin_add.png}
	\caption{Manager Add dish Page}
\end{figure}
\paragraph{Add Dish Page}
In the Add Dish Page is shown a form whith an empty text input for the new dish  Name and a Price input that can either be typed or increased and deacresed through the specific buttons. The new dish is added by clicking ``Submit''.
It is possible for the manager to go back to the Menu Operation page by clicking ``Back''.

\subsubsection{Accessing the Orders Operations}
\begin{figure}[H]
	\centering
	\includegraphics[width=7cm]{../../documenti/UserManualDemo/demo_screens/admin_main.png}
	\caption{Manager Main Page}
\end{figure}
\paragraph{Manager Main Page}
In the Manager Main Page is possible to access the Orders Operations page by clicking the button labeled as ``Orders Operations''.
